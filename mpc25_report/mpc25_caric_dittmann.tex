\documentclass[10pt,twocolumn,twoside]{IEEEtran}

\IEEEoverridecommandlockouts    
\overrideIEEEmargins

\usepackage{psfrag}
\usepackage{amsmath,amssymb,amsfonts,bbm,nicefrac,mathtools,lipsum}
\usepackage{latexsym}
\usepackage{graphicx}


\usepackage{balance}
\usepackage{colortbl}
\usepackage{fancyhdr}
\usepackage{epstopdf}
\usepackage{float}
\usepackage{hyperref}
\usepackage{booktabs,tabularx}

%\RequirePackage{fix-cm}
%\makeatletter 
%\newcommand{\semiHuge}{\@setfontsize\semiHuge{22 }{27.38}}
%\makeatother


%\usepackage{tikz}
%\usepackage{pgfplots}
%\usetikzlibrary{chains}
%\usetikzlibrary{fit}
%\usepgfplotslibrary{dateplot}
%\usetikzlibrary{shapes.symbols,patterns, arrows} % for source symbols

%\newcommand{\ds}{\displaystyle}
%\newcommand{\qq}{\quad\quad}
%\newcommand{\Var} {\mathrm{Var}\,}
%\newcommand{\Cov} {\mathrm{Cov}\,}

%\newtheorem{theorem}{Theorem}
%\newtheorem{definition}{Definition}
%\newtheorem{proposition}{Proposition}
%\newtheorem{lemma}{Lemma}
%\newtheorem{corollary}{Corollary}
%\newtheorem{example}{Example}
%\newtheorem{remark}{Remark}
%\newtheorem{assumption}{Assumption}
%\newtheorem{standing}{Standing Assumption}
%\newtheorem{design}{Design choice}


\usepackage{soul}
\usepackage[ruled]{algorithm2e}

% Commands
\newcommand{\R}{\mathbb{R}}
\newcommand{\Z}{\mathbb{Z}}
\newcommand{\N}{\mathbb{N}}
\newcommand{\X}{\mathbb{X}}
\newcommand{\U}{\mathbb{U}}
\newcommand{\mc}{\mathcal}
\newcommand{\A}{\mc{A}}
\newcommand{\bx}{\mathbf{x}}
\newcommand{\bu}{\mathbf{u}}
\newcommand{\bbS}{\mathbb{S}}

\newcommand{\prox}{\mathrm{prox}}
\newcommand{\fix}{\mathrm{fix}}
\newcommand{\zer}{\mathrm{zer}}
\newcommand{\Id}{\mathrm{Id}}
\newcommand{\dist}{\mathrm{dist}}
\newcommand{\argmin}{\mathrm{argmin}}
\newcommand{\dom}{\mathrm{dom}}
\newcommand{\proj}{\mathrm{proj}}
\newcommand{\diag}{\mathrm{diag}}
\newcommand{\gph}{\mathrm{gph}}


\newcommand{\bs}{\boldsymbol}
\def\bsu{\boldsymbol{u}}
\def\bsx{\boldsymbol{x}}
\def\ni{\noindent}
\def\xx{\chi}
\def\ub{\textbf{u}}
\def\zb{\textbf{z}}

\newcommand{\norm}[1]{\left\|#1\right\|}

\newcommand{\red}{\textcolor{red}}
\newcommand{\blue}{\textcolor{blue}}


%\newcommand*{\defeq}{\mathrel{\vcenter{\baselineskip0.5ex \lineskiplimit0pt
%\hbox{\scriptsize.}\hbox{\scriptsize.}}}%
%=}


\graphicspath{{./figures/}}

\date{ }

\begin{document}

\title{A model predictive control approach for Some Application...}

\author{Name Surname 1, Name Surname 2, 
\thanks{Name Surname 1 and Name Surname 2 are master students at TU Delft, The Netherlands. 
E-mail addresses: \{\texttt{n.surname1, n.surname2}\}\texttt{@student.tudelft.nl}. \smallskip
}
}
\maketitle         

% How to find "Some Application":
% 1. http://ieeexplore.ieee.org
% 2. Search "model predictive control" from 201x on
% 3. Restrict search to "Journals & Magazines"
% 4. Restrict search to the "Publication Title" (titles) of interest (e.g. Intelligent Transportation Systems, Power Systems, Power Electronics, ...)


\begin{abstract}
We study a model predictive control (MPC) approach for Some Application....
\end{abstract}

\section{Introduction} \label{sec:intro}
Some Application is ... relevant for... The standard control strategy for Some Application is... which... cannot handle operational constraints... and/or may result in... non-optimal performance...

The systems dynamics are linear/nonlinear, ... where ... are the system states, ... are the control inputs, and ... are the system outputs.

\section{Model predictive control design}

We consider a receding horizon MPC strategy with control horizon $N=...$

The state constraints are... hence can be represented in compact form as $F x(k) \leq e$, where $F$ is the matrix 
$$ F = \left[
\begin{matrix}
* & * & 0 \\
* & * & * \\
* & * & *
\end{matrix}
\right]
 $$
and $e$ is the vector...

Control input constraints...

Output constraints...

We design the stage cost function $\ell(x,u) = ...$, the terminal cost function $V_{\textup{f}}(x) = ...$, and the terminal set $\mathbb{X}_{\textup{f}} = ...$

\section{Asymptotic stability}

In this section, we show that the designed MPC asymptotically stabilized the closed-loop system. With this aim, we verify the assumptions of Theorem... in the book \cite{rawlings:mayne}

Assumption 2.x: ... 

Assumption 2.y: ...


\section{Numerical simulations}

In this section, we run several numerical simulations where we compare some MPC controllers as well as standard controllers used in Some Application.

Since in Some Application, the state/output cannot be accurately measured, we assume the presence of a random measurement disturbance...

Plots... show the effect of shorter/longer control horizon... tuning the cost matrices... Compared with Standard Control, we observe that...

\bibliographystyle{IEEEtran}

\bibliography{mpc_library}

\end{document}
